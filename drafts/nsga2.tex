\documentclass[a4paper, 10pt]{article}

% Nécessaire
\usepackage[french]{babel}
\usepackage[utf8]{inputenc}
\usepackage[T1]{fontenc}
\usepackage{lmodern}
\usepackage{amsmath, amsthm}
\usepackage{amsfonts,amssymb}

% Marge
\usepackage{geometry}
\geometry{margin={2.2cm ,2cm}}

% Figures, graphiques
\usepackage{graphicx}
\usepackage{epsfig}
\usepackage{caption}

% Surlignage
\usepackage{alltt}

\usepackage{xcolor}
\usepackage{soul}
\usepackage{color}
\usepackage{colortbl}

% Indicatrice
\usepackage{dsfont}

\usepackage{multirow}
\usepackage{eurosym}
\usepackage{extarrows}
% \usepackage{pdfpages}


% Titre
\title{Optimisation multi-objectifs avec l'algorithme NSGA-II}
\author{}
\date{}


\begin{document}
\maketitle

La résolution d'un problème à plusieurs objectifs n'a que très rarement une unique solution. Les algorithmes d'optimisation de problèmes multi-objectifs, dont NSGA-II tel que décrit dans \cite{nsga}, propose  alors un sous-ensemble de solutions inclus dans l'optimum de Pareto. L'optimum de Pareto (aussi appelé ensemble de Pareto) d'un problème est un ensemble regroupant toutes les solutions non-dominées d'un problème. Est qualifiée non-dominée une solution $x\in \mathbb{R}^p$ pour laquelle il n'existe pas de solution $y$ telle que $y_1 \succ x_1, \ldots, y_p \succ x_p$.




\bibliographystyle{alpha} 
\bibliography{nsga}


\end{document}


